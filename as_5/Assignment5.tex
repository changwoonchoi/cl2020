\documentclass[12pt]{article}%
\usepackage{amsfonts}
\usepackage{fancyhdr}
\usepackage{comment}
\usepackage[a4paper, top=2.5cm, bottom=2.5cm, left=2.2cm, right=2.2cm]%
{geometry}
\usepackage{times}
\usepackage{amsmath}
\usepackage{changepage}
\usepackage{amssymb}
\usepackage{graphicx}%
\setcounter{MaxMatrixCols}{30}
\newtheorem{theorem}{Theorem}
\newtheorem{acknowledgement}[theorem]{Acknowledgement}
\newtheorem{algorithm}[theorem]{Algorithm}
\newtheorem{axiom}{Axiom}
\newtheorem{case}[theorem]{Case}
\newtheorem{claim}[theorem]{Claim}
\newtheorem{conclusion}[theorem]{Conclusion}
\newtheorem{condition}[theorem]{Condition}
\newtheorem{conjecture}[theorem]{Conjecture}
\newtheorem{corollary}[theorem]{Corollary}
\newtheorem{criterion}[theorem]{Criterion}
\newtheorem{definition}[theorem]{Definition}
\newtheorem{example}[theorem]{Example}
\newtheorem{exercise}[theorem]{Exercise}
\newtheorem{lemma}[theorem]{Lemma}
\newtheorem{notation}[theorem]{Notation}
\newtheorem{problem}[theorem]{Problem}
\newtheorem{proposition}[theorem]{Proposition}
\newtheorem{remark}[theorem]{Remark}
\newtheorem{solution}[theorem]{Solution}
\newtheorem{summary}[theorem]{Summary}
\newenvironment{proof}[1][Proof]{\textbf{#1.} }{\ \rule{0.5em}{0.5em}}

\newcommand{\Q}{\mathbb{Q}}
\newcommand{\R}{\mathbb{R}}
\newcommand{\C}{\mathbb{C}}
\newcommand{\Z}{\mathbb{Z}}

\begin{document}

\title{Assignment 5}
\author{Changwoon Choi \\ 2020-20206}
\date{\today}
\maketitle

\section{Logistic Regression}
Suppose you train a logistic regression classifier and your hypothesis function is $H_\theta (x) = g(\theta_0 + \theta_1 x_1 + \theta_2 x_2 ).$ Where $g$ is logistic function and $\theta_0 , \theta_1 ,$ and $\theta_2$ are 6, 0, and -1, respectively. Draw the decision boundaries for classification and indicate where the positive prediction is.
\\

Ans) 

\section{Logistic function}
The logistic function is defined as $g(z) = {1 \over 1 + e^{-z}}$. It is known that the logistic function is easy to differentiate because it has a simple derivative $ {\partial \over \partial z}g(z) = g(z)(1 - g(z)).$ Derive this equation.\\

Ans)

\section{Gradient descent for logistic regression}
Using the equation above, derive the gradient descent algorithm for logistic regression.\\

Ans)

\section{Cost function for logistic regression}
Derive the cost function $J(\theta)$ from the perspective of the maximum (log) likelihood; i.e. by maximizing the joint probability of $m$ training samples being correctly classified. (Hint: the probability of $y=1$ or $0$, given $x$, parameterized by $\theta$ can be written more compactly as $P(y | x; \theta) = (h_\theta (x))^y (1 - h_\theta (x))^{1-y} )$)\\

Ans)
\end{document}